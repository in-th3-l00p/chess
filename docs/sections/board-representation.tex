%! Author = intheloop
%! Date = 3/29/25

\section{Board representation}

\textbf{Board representation} is stored using an array with a \textit{12x10} dimension of \textit{1 byte words}.

\subsection{Piece representation}
\vspace{0.02\linewidth}
\begin{center}
    \begin{bytefield}[
        endianness=little,
        bitwidth=0.1\linewidth,
        boxformatting={\centering\small}
    ]{8}
        \bitheader{0-7} \\
        \bitbox{1}{Color}
        \bitbox{1}{}
        \bitbox{1}{}
        \bitbox{1}{Castle}
        \bitbox{1}{Moved}
        \bitbox{1}{Type}
        \bitbox{1}{Type}
        \bitbox{1}{Type}
    \end{bytefield}
\end{center}
\begin{itemize}
    \item Color: \textbf{1} - Black, \textbf{0} - White
    \item Castle \scriptsize(king only)\normalsize: \textbf{1} - can castle, \textbf{0} - cannot
    \item Moved \scriptsize(king, rook, and pawn only)\normalsize: \textbf{1} - has moved, \textbf{0} - not yet
    \item Type: refer to \textbf{Piece type} subsection
\end{itemize}

\subsection{Piece type}

The last 3 bits of a \textit{word} representing a piece:

\begin{itemize}
    \item \textbf{001} - Pawn
    \item \textbf{010} - Knight
    \item \textbf{011} - Bishop
    \item \textbf{100} - Rook
    \item \textbf{101} - Queen
    \item \textbf{110} - King
\end{itemize}

There are 2 more special \textit{words} used:

\begin{itemize}
    \item \textbf{000} - empty \scriptsize(pieces can move here)\normalsize
    \item \textbf{255} - invalid \scriptsize(sentinel value)\normalsize
\end{itemize}